\chapter{Method}
In this chapter the work environment and the procedure is explained. 

\section{Frameworks and tools}
This work uses Python $3.8.10$ as programming language for the data analysis as well as for the training and evaluation of the models. Python already offers many high-end libraries for \gls{ml}. Jupyter notebooks were used to write and adjust all code. Important snippets were moved into python files to reuse it in multiple notebooks.

\section{Public plant disease datasets}
\label{section:plant_datasets}

The following Table \ref{tab:suitable_plant_datasets} lists a subset of suitable plant disease datasets. The complete table including links is in the appendix \ref{appendix:datasets_tables}.

the various publicly available datasets with plant disease images. 

\begin{table}[H]
\centering
\caption{Large plant disease datasets \label{tab:suitable_plant_datasets}}
\begin{tabularx}{\textwidth}{|
 >{\hsize=.72\hsize}X |
 >{\hsize=.14\hsize\raggedleft}X |
 >{\hsize=.14\hsize\raggedleft}X |
}
\hline
\textbf{Name} & \textbf{\#Images} & \textbf{\#Classes} \tabularnewline \hline
PlantVillage (PVD) \autocite{hughes2016} & 54'303 & 38 \tabularnewline \hline
Cassava Leaf Disease Classification \autocite{mwebaze2020} & 21'398 & 5 \tabularnewline \hline
PlantDataset & 5'106 & 20 \tabularnewline \hline
PlantDoc \autocite{singh2020} & 2'598 & 28 \tabularnewline \hline
DARMA \autocite{keaton2021} & 231'414  & 1'000 \tabularnewline \hline
Plant disease diagnosis dataset (PDDD) \autocite{dong2023} & 421'133  & 120 \tabularnewline \hline
\end{tabularx}
\end{table}


\subsection{...}

\section{Models}
