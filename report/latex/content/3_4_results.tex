\chapter{Results}

% \subsection{Linear evaluation on features}
\section{Overview}
\subsection{Resnet}

The following Figure~\ref{fig:spider_resnet50} shows the result with Resnet50, while Table~\ref{tab:f1_scores_resnet_derma} and Table~\ref{tab:f1_scores_resnet_plant} contain the respective values.
\begin{figure}[H]
    \begin{center}
    \includegraphics[width=15cm]{spider_resnet50.png}
    \caption{Linear evaluation of Resnet50}\label{fig:spider_resnet50}
    \end{center}
\end{figure}

The first thing to notice is the awful performance of the model trained on dermatology with SimCLR. The scores are never significantly better than the other pre-trained models. The model which was trained on ImageNet with \gls{ssl} was not put into the graphic, but the tables show that these scores are bad as well. 
This indicates a problem with the type of training rather than with the data per se.

In the linear evaluation the plant based model is actually a bit better than the ImageNet based model, but with \gls{knn} the difference becomes negligible.



\subsection{Vision Transformers}

The results of the \gls{vit} can bee seen in Figure~\ref{fig:spider_vit_t16}, Table~\ref{tab:f1_scores_vit_derma} and Table~\ref{tab:f1_scores_vit_plant}. 
The scores look similar. The scores of the plant based model are better in some tasks than the model trained with \gls{sl} on ImageNet, but not better than the \gls{ssl} variant. All DINO based models are relatively close to each other. This is probably due to their way of training rather than the data.

\begin{figure}[H]
    \begin{center}
    \includegraphics[width=15cm]{spider_vit_t16.png}
    \caption{Linear evaluation of \gls{vit}}\label{fig:spider_vit_t16}
    \end{center}
\end{figure}


\section{Subsampling}
% NOTE: order by size
\subsection{PlantDoc}

\begin{figure}[H]
    \begin{center}
    \includegraphics[width=15cm]{lines_plantdoc.png}
    \caption{Results of linear evaluation on subsampled sets from PlantDoc}\label{fig:lines_plantdoc}
    \end{center}
\end{figure}

\subsection{PlantDataset}

\begin{figure}[H]
    \begin{center}
    \includegraphics[width=15cm]{lines_plantdataset.png}
    \caption{Results of linear evaluation on subsampled sets from PlantDataset}\label{fig:lines_plantdataset}
    \end{center}
\end{figure}

\subsection{Cassava Leaf Disease Classification}

\begin{figure}[H]
    \begin{center}
    \includegraphics[width=15cm]{lines_cassava.png}
    \caption{Results of linear evaluation on subsampled sets from Cassava}\label{fig:lines_cassava}
    \end{center}
\end{figure}

\subsection{PlantVillage Dataset (PVD)}

\begin{figure}[H]
    \begin{center}
    \includegraphics[width=15cm]{lines_plantvillage.png}
    \caption{Results of linear evaluation on subsampled sets from \gls{pvd}}\label{fig:lines_plantvillage}
    \end{center}
\end{figure}

\subsection{Diverse Dermatology Images (DDI)}

\begin{figure}[H]
    \begin{center}
    \includegraphics[width=15cm]{lines_ddi.png}
    \caption{Results of linear evaluation on subsampled sets from \gls{ddi}}\label{fig:lines_ddi}
    \end{center}
\end{figure}

\subsection{PAD-UFES-20}

\begin{figure}[H]
    \begin{center}
    \includegraphics[width=15cm]{lines_pad-ufes-20.png}
    \caption{Results of linear evaluation on subsampled sets from PAD-UFES-20}\label{fig:lines_pad-ufes-20}
    \end{center}
\end{figure}

\subsection{HAM10000}

\begin{figure}[H]
    \begin{center}
    \includegraphics[width=15cm]{lines_ham10000.png}
    \caption{Results of linear evaluation on subsampled sets from HAM10000}\label{fig:lines_ham10000}
    \end{center}
\end{figure}

\subsection{Fitzpatrick17k}

\begin{figure}[H]
    \begin{center}
    \includegraphics[width=15cm]{lines_fitzpatrick17k.png}
    \caption{Results of linear evaluation on subsampled sets from Fitzpatrick17k}\label{fig:lines_fitzpatrick17k}
    \end{center}
\end{figure}


% \section{Other}
% \subsection{DINO student/teacher}
% \subsection{Fine-tuning}
% \subsection{Fine-tuning}
